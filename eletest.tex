\documentclass[cn,11pt,chinese]{elegantbook}
\usepackage[version=4]{mhchem}
\title{现代VLSI器件基础}
\subtitle{fundamentals of modern VLSI devices 译本}

\author{YUAN TAUR \& TAK H.NING}
\institute{UCSD\& IBM}
\date{\today}
\bioinfo{译者}{郑博航}

\extrainfo{Victory won\rq t come to us unless we go to it. --- M. Moore}

\logo{logo-blue.png}
\cover{cover.jpg}

% 本文档命令
\usepackage{array}
\newcommand{\ccr}[1]{\makecell{{\color{#1}\rule{1cm}{1cm}}}}
% 修改目录深度
\setcounter{tocdepth}{2}

\begin{document}
	
	\maketitle
	\frontmatter
\chapter*{第二版前言}
\markboth{Introduction}{前言}
自从剑桥大学在1998年出版了第一版的现代VLSI器件基础,我们收到了关于此书的许多赞誉和鼓励。许多美国和全世界各地的重点高校将它作为研究生一年级有关微电子课程的教科书。在2002年,广岛大学Shibahara教授所领导的团队已经将第一版翻译成了日文。

在过去的十年中,VLSI(超大规模集成电路)技术的进步和发展在继续着。如今,晶体管的发明已经经过了60年,每个处理器或者DRAM芯片上已经集成了超过十亿个晶体管,并且最高的处理器时钟频率已经高达5GHz。在2007年,全世界的集成电路规模高达2500亿美元。在2008年时,IC工业界到达了45纳米的工艺节点,这意味着尖端的IC产品采取了最小45纳米的光刻特征尺寸。当bulk CMOS工艺被缩减到低于100nm的尺寸时,让CMOS技术成为制造大规模数字集成电路的特殊因素,也就是低静态功耗,不能再被看作是理所应当的了。虽然关态的电流随着现在已经减小到1V的源电压增加而增加,但是由于栅极氧化层只有几个原子层的厚度,栅极的泄露电流会因为量子隧穿效应指数倍增加。动态和静态的电源管理已经变成了继续在处理器里提高时钟频率和晶体管数量的关键因素。为了继续将尺寸减小到10nm,新的材料和器件结构已经被探索去替代传统的bulk CMOS。

写第二版书的目的是为了更新在第一版完成后在材料上的进步。关键的新材料被新增,诸如MOSFET按比例缩小理论(MOSFET scale length theory),高场传输模型(high-field transport model)和SiGebase双极型器件那一章已经被极大的增加了。我们也增加了关于基础器件物理和电路的讨论,包括金半接触(metal-silicons contacts),COMS电路的噪声容限和射频应用优点的图像。此外第二版还增加了两个全新的章节。第九章是关于存储器件并且包括常用的SRAM,DRAM的读写操作基础和非易失存储器阵列。第十章是关于SOI(绝缘底上硅)器件包括先进器件的未来潜力。

我愿意借此机会感谢所有在本书的进步中有给予我们鼓励和有价值的建议的朋友和同事。特别是很早就采用了第一版的普渡大学的Mark Lundstrom教授和给出关于处理扩散电容建议的国家半导体公司的Constantin Bulucea博士。同时感谢乔治亚理工学院的James Meindl教授,UCSD的Peter Asbeck教授和佛罗里达大学的Jerry Fossum教授对本书的支持。

我们也要感谢在IBM的许多同事,特别是在先进的硅器件研发领域的,感谢他们直接或间接的贡献。Yuan Taur也要感谢他在UCSD的许多学生对完成第二版的帮助,特别是Jooyoung Song和Bo Yu。我也要感谢Katie Kahng在我工作过程中的中的爱,支持和耐心。

我们要特别感谢我们的家人在这个看似无穷无尽的任务中对我们的支持理解。
\vskip 1.5cm

\begin{flushright}
	Yuan Taur\\
	Tak H.Ning\\
    June,2008
\end{flushright}

\tableofcontents
%\listofchanges
\mainmatter
\chapter{引言}
自从1947年双极型晶体管被发明,半导体工业就得到了空前的增长,并且对人们的工作生活带来了巨大的影响。大约在最近的30年,到目前为止半导体发展最快的领域是硅基的超大规模集成电路技术,晶体管尺寸的持续缩小是VLSI技术持续发展的驱动。小型化的益处,比如更高的封装密度,更高的电路速度,以及更低的功耗,是带来如今革命性进步的关键,这些进步包括与上一代比起来,如今的计算机,无线单元和通信系统拥有的的更高性能,每个功能模块戏剧性减少的消耗,以及大大减少的物理尺寸。在经济这一方面,整个世界的IC商业规模由1970年的10亿美元,到1984年的200亿美元,2007年已经到达了2500亿美元。就产出和就业来说,电子工业现在在许多国家已经跻身于最大的工业品类之列。在全世界范围内,微电子对于经济社会甚至政治的重要性无疑将继续增加。世界范围内在VLSI技术的投资成为了强有力的驱动力,那几乎保障了在物理定律允许的情况下IC集成的密度和功耗的持续进步。

\section{VLSI器件工艺的发展}
\subsection{历史展望}
在1980年中期,Sah的一篇文章叙述了一种绝好的关于mosfet的改进,形成了它在VLSI应用的初步概念。图1.1给出了在VLSI技术发展具有里程碑意义时间的年表。二极管技术在早期被发展并且被1960年代被首先用于大型主机的集成内存。双极型晶体管一直被用在原始器件速度最重要的时候,因为双极型电路在独立电路等级保持着最快速度。但是,当单位芯片到达大约$10^4$个电路时,双极型器件的巨大功耗严重限制了他们的集成度。在如今的VLSI标准看来,这样的集成度是非常低的。
%此处插入figure1.1

1930年第一次发表了通过应用电场调制半导体表面电导的想法。然而,早期尝试去制造一个表面电场控制的器件并不成功,原因在于高密度的表面态的存在有效地防止了外部电场影响的表面电势。第一块基于硅衬底并且栅极绝缘层为\ce{SiO2}的MOSFET于1960年被制造(Kahng and Atalla,1960)。为了在一块 硅芯上实现电路功能,在1960,1970年代,n沟道和p沟道的MOSFET以及双极型晶体管被广泛的使用。尽管MOSFET比双极型器件要慢,但是它有更高的布局密度并且制造也相对简单。最简单的MOSFET芯片可以仅用四张掩膜和单次参杂来完成。然而,就像双极型电路,单极的MOSFET电路也受着大静态功耗之苦,并且因此单个芯片上的集成度是有限的。


在1963年,COMS(互补型MOS)的发明(Wanlass and Sah)使得集成度取得了重大的突破,此时n沟道和p沟道的MOSFET在同一块衬底上被并排构建。在一个典型的COMS电路中,n沟道和p沟道的MOSFET在电源端被串联,所以说静态功耗可以说是微不足道的。绝大部分的功耗被消耗在电路的开关状态(比如仅当在动态电路时)。通过清楚地在一块电路上设计电路的“开关活动”,工程师已经可以在一块芯片上集成上亿个CMOS晶体管并且使得芯片容易地空气冷却。直到最小的光刻特征尺寸到达180nm,CMOS的集成水平不再受功耗的限制,而取决于制造工艺。另一个CMOS电路的优点来自于无比例,全轨到轨摆幅\footnote{原文为ratioless,full rail-to-rail logic swing限于译者水平,翻译的不好},这可以增加噪声容限,使得设计一个CMOS芯片变得更容易。

%此处插入figure1.2

随着线性尺寸在1990年代早期到达$0.5-\mu m$尺寸的水平,双极型晶体管的优点已经被CMOS器件巨大的电路密度所超过了。集成功能的系统性优势取代了原始的晶体管性能。即便是高端的电脑系统设计者也可以用CMOS来取代双极型二极管($Rao et al.,1997$)以满足他们的性能目标。从此,CMOS已经成为了数字电路的技术,并且双极的仅仅主要被用于模拟和射频。



\end{document}